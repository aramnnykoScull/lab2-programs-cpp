% !TeX program = xelatex

% Явное указание компилятора для редакторов
\documentclass[14pt, a4paper]{extarticle}

\usepackage{fontspec}
\usepackage{polyglossia}
\setdefaultlanguage{russian}
\setotherlanguage{english}

% Основной шрифт для всего документа
\setmainfont{Times New Roman}
% Явное указание кириллического шрифта
\newfontfamily\cyrillicfont{Times New Roman}
% Шрифт для моноширинного текста (листинги, код) с поддержкой кириллицы
\newfontfamily\cyrillicfonttt{Times New Roman} % Если хотите везде Times New Roman

% Пакеты для оформления и содержания 
\usepackage{geometry}
\usepackage{titlesec}
\usepackage{setspace}
\usepackage{graphicx}
\usepackage{indentfirst}
\usepackage{enumitem}
\usepackage{float}
\usepackage{listings}
\usepackage{caption}
\usepackage{hyperref}
\usepackage{ragged2e}
\usepackage{needspace}

\usepackage{needspace}
\usepackage{etoolbox}

% Настройки для предотвращения плохих разрывов листингов
\BeforeBeginEnvironment{lstlisting}{%
  \par
  \needspace{7\baselineskip}% Минимум 7 строк должно остаться
  \nopagebreak[4]% Максимальный приоритет запрета разрыва
  \penalty -500\relax% Поощряем разрыв ДО листинга если нужно
}

\AfterEndEnvironment{lstlisting}{%
  \par
  \nopagebreak[3]% Средний приоритет запрета разрыва после
  \penalty 3000\relax% Штрафуем разрыв сразу после рамки
}

% Глобальные настройки для борьбы с "висячими" строками
\clubpenalty=10000    % Штраф за одиночную строку в начале страницы
\widowpenalty=10000   % Штраф за одиночную строку в конце страницы
\brokenpenalty=10000  % Штраф за разбитую строку с переносом

% Улучшаем настройки листингов
\lstset{
    language=C++,
    basicstyle=\ttfamily\footnotesize,
    numbers=left,
    numberstyle=\tiny,
    frame=single,
    breaklines=true,
    breakatwhitespace=true,
    showspaces=false,
    showstringspaces=false,
    captionpos=b,
    inputencoding=utf8,
    extendedchars=true,
    keepspaces=true,
    % Увеличиваем вертикальные отступы
    aboveskip=10pt,
    belowskip=10pt,
    % Индикаторы переноса
    prebreak=\mbox{\quad$\hookrightarrow$},
    postbreak=\mbox{$\hookleftarrow$\space},
    % Улучшаем перенос длинных строк
    tabsize=4,
    columns=flexible,
}


% Настройки по ГОСТ 
\geometry{left=30mm, right=15mm, top=20mm, bottom=20mm}
\onehalfspacing
\parindent=1.25cm
\pagestyle{plain} % Добавляем нумерацию страниц внизу по центру (по ГОСТ)

% Настройка стилей заголовков по ГОСТ 
\titleformat{\section}{\normalfont\bfseries\centering}{\thesection}{1em}{}
\titleformat{\subsection}{\normalfont\bfseries\raggedright}{\thesubsection}{1em}{}
\titleformat{\subsubsection}{\normalfont\bfseries\raggedright}{\thesubsubsection}{1em}{}


% Настройка нумерации листингов 
\renewcommand{\lstlistingname}{Листинг}
\renewcommand{\lstlistlistingname}{Список листингов}

% Настройка списков 
\setlist[itemize]{leftmargin=*}
\setlist[enumerate]{leftmargin=*}

% Выравнивание текста по ширине 
\justifying

\begin{document}  


\subsection*{Цель работы:}
Изучить разработку и реализацию программ линейной структуры, освоить базовые операции ввода-вывода и простые вычисления.

\subsection*{Задание:}
\begin{enumerate}[label=\arabic*)]
    \item Разработать консольное приложение для подсчета площади квадрата. Пользователь вводит сторону и получает сообщение о полученной площади.
    \item Разработать консольное приложение для подсчета объема прямоугольного параллелепипеда. Пользователь вводит три стороны и получает сообщение о полученном объеме.
    \item Разработать консольное приложение для подсчета пройденного расстояния. Пользователь вводит скорость (в км/час) и время (в часах) и получает сообщение о пройденном расстоянии (в км).
\end{enumerate}

\subsection*{Ход работы:}

\subsubsection*{Теоретическая информация}
Линейные программы выполняют последовательность операций без ветвлений и циклов. В данной работе реализованы простые вычисления с использованием функций для безопасного ввода данных. Основное внимание уделяется обработке пользовательского ввода, преобразованию типов и базовым арифметическим операциям.

\subsubsection*{Подзадание №1:}
Программа вычисляет площадь квадрата по введенной стороне. Используется функция для чтения и проверки целого числа. При вводе отрицательного или некорректного значения программа выводит сообщение об ошибке.
Код программы представлен ниже (Листинг \ref{lst:lab2_point1}, \ref{lst:lab2_point_all_h} и \ref{lst:lab2_point_all_cpp}, Рисунок \ref{fig:lab2_image1}).

\begin{lstlisting}[caption={Код программы для задания 1}, label=lst:lab2_point1]
#include "point_all.h"

int main() {
    int side;

    if (!read_int("Введите сторону квадрата: ", &side)) {
        return 1;
    }

    cout << "Площадь квадрата: " << side * side << endl;

    return 0;
}
\end{lstlisting}

\begin{figure}[H]
    \centering
    \includegraphics[width=0.4\textwidth, height=0.4\textheight, keepaspectratio]{media/lab2_image1.png}
    \caption{Пример выполнения программы для задания 1}
    \label{fig:lab2_image1}
\end{figure}

\subsubsection*{Подзадание №2:}
Программа вычисляет объем прямоугольного параллелепипеда по трем сторонам. Реализован ввод и проверка данных. Для вычисления объема используется формула \(V = a \times b \times h\).
Код программы представлен ниже (Листинг \ref{lst:lab2_point2}, \ref{lst:lab2_point_all_h} и \ref{lst:lab2_point_all_cpp}, Рисунок \ref{fig:lab2_image2}).

\begin{lstlisting}[caption={Код программы для задания 2}, label=lst:lab2_point2]
#include "point_all.h"

int main() {
    int side_a, side_b, side_h;

    cout << "Для подсчета объема прямоугольного параллелепипеда введите три стороны: " << endl;
    
    if (!read_int("Введите сторону a (длина): ", &side_a) ||
        !read_int("Введите сторону b (ширина): ", &side_b) ||
        !read_int("Введите сторону h (высота): ", &side_h)) {
        return 1;
    }
    
    cout << "Объем прямоугольного параллелепипеда: " << side_a * side_b * side_h << endl;

    return 0;
}
\end{lstlisting}

\begin{figure}[H]
    \centering
    \includegraphics[width=0.7\textwidth, height=0.7\textheight, keepaspectratio]{media/lab2_image2.png}
    \caption{Пример выполнения программы для задания 2}
    \label{fig:lab2_image2}
\end{figure}

\subsubsection*{Подзадание №3:}
Программа вычисляет пройденное расстояние по скорости и времени. Используется функция для чтения чисел с плавающей точкой. Расстояние вычисляется по формуле \(S = v \times t\).
Код программы представлен ниже (Листинг \ref{lst:lab2_point3}, \ref{lst:lab2_point_all_h} и \ref{lst:lab2_point_all_cpp}, Рисунок \ref{fig:lab2_image3}).

\begin{lstlisting}[caption={Код программы для задания 3}, label=lst:lab2_point3]
#include "point_all.h"

int main() {
    int speed, time;

    cout << "Для подсчета пройденного расстояния введите скорость (в км/час) и время (в часах): " << endl;
    
    if (!read_int("Введите скорость (только число): ", &speed) ||
        !read_int("Введите время (только число): ", &time)) {
        return 1;
    }
    
    cout << "Пройденное расстояние: " << speed * time << " км" << endl;

    return 0;
}
\end{lstlisting}

\begin{figure}[H]
    \centering
    \includegraphics[width=0.8\textwidth, height=0.8\textheight, keepaspectratio]{media/lab2_image3.png}
    \caption{Пример выполнения программы для задания 3}
    \label{fig:lab2_image3}
\end{figure}

\subsubsection*{Вспомогательные файлы:}
Для всех программ использован общий заголовочный файл и функция для чтения и проверки положительного числа.
\textit{(Данная библиотека будет использоваться в последующих лабораторных работах)}

\begin{lstlisting}[caption={Заголовочный файл point\_all.h}, label=lst:lab2_point_all_h]
#ifndef POINT_ALL_H
#define POINT_ALL_H

#include <iostream>
#include <string>
using namespace std;

// Функция для чтения и проверки положительного числа
bool read_int(const string& prompt, int* value);

#endif
\end{lstlisting}

\begin{lstlisting}[caption={Файл реализации point\_all.cpp}, label=lst:lab2_point_all_cpp]
#include "point_all.h"

bool read_int(const string& prompt, int* value) {
    string input;
    cout << prompt;
    
    if (!getline(cin, input) || input.empty()) {
        cout << "Вы не ввели число" << endl;
        return false;
    }
    
    try {
        *value = stoi(input);
        if (*value <= 0) {
            cout << "Вы ввели некорректное число" << endl;
            return false;
        }
        return true;
    } catch (...) {
        cout << "Вы ввели не число" << endl;
        return false;
    }
}
\end{lstlisting}

\subsection*{Вывод:}
В ходе работы были изучены принципы разработки программ линейной структуры. Реализованы приложения для вычисления площади, объема и расстояния с использованием функций безопасного ввода. Приобретены навыки работы с базовыми типами данных, вводом-выводом и простыми арифметическими операциями.

\end{document}